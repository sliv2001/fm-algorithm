% !TeX spellcheck = en_US
\documentclass[12pt,a4paper,twocolumn]{article}
\usepackage[utf8]{inputenc}
\usepackage[T1]{fontenc}
\usepackage{multirow}
\usepackage[left=3.00cm, right=1.00cm, top=2.00cm, bottom=2.00cm]{geometry}
\usepackage{amssymb}
\usepackage{graphicx}
\usepackage{multicol}
\title{Fiduccia-Mattheyses algorithm. Implementation and modification}
\author{Ivan Sladkov, B01-908, MIPT}
\begin{document}
	\maketitle
	
	Being an heuristic for solution of a problem of graph partition, Fiduccia-Mattheyses algorithm does not give a completely precise solution, though resulting partition converges to it. Some modifications may be introduced in order to increase convergence speed or precision of partition. In this work original algorithm was implemented and a modification was added to increase speed of algorithm.
	
	\section{On algorithm implementation}
	
	The original algorithm was implemented with no clustering and with accent on balance. For a formula  
	\begin{equation}\label{eq:usualBalance}
		r W - S_{max} \le \left| A \right| \le r W +S_{max}
	\end{equation}
	from %source
	, which describes soft balancing, coefficient $r$ is taken as $0.5$, while maximum divergence between $| A |$ and $| B |$ is $S_{max} = 1$.
	
	Implementation was made in C++. All the containers used in project are based on Standard Template Library. The data structures used are as follows:
	\begin{itemize}
		\item \texttt{std::map<int, std::list<int> > gc} -- gain bucket 
		\item \texttt{std::vector<bool> erased} -- shows whether vertex is in bucket or not
		\item \texttt{std::vector<std::pair<int, iterator> > searchSupport} -- structure to facilitate and accelerate search
	\end{itemize}
	Search through gain buckets is made as $O(1)$.
	
	\section{Modification of algorithm}
	
	In order to make algorithm converge faster a <<cutoff>> modification was implemented: if movement of a vertex gives significant cost increase, then further motion is not done. The original idea was to avoid applying of vertices movement, whose gain is negative. Then the threshold was added. So function \texttt{FMPass()} is transformed into:
	\begin{verbatim}
		function FMpass
		(gain_container, partitionment) :
		
		solution_cost = 
		partitionment.get cost()
		
		while not all vertices locked {
		
		move = best_feasible_move()
		
		solution_cost-= 
		gain_container.get gain (move)
		
		if (solution_cost > best_cost+threshold)
		then break
		
		gain_container.lock_vertex(
		move.vertex())
		
		gain_update (move)
		
		partitionment.apply (move)
	}
	
	roll back partitionment 
	to best seen solution
	
	gain_container.unlock_all()

	\end{verbatim}
	
	\section{Comparison}
	
	All benchmarks were executed on Apple M2 processor. Result of comparison is shown in table \ref{tab:res}.
	% Please add the following required packages to your document preamble:
	% \usepackage{multirow}
	\begin{table*}[tb]
		\onecolumn
		\small
\begin{tabular}{|lllll|l|l|}
	\hline
	\multicolumn{1}{|l|}{\multirow{2}{*}{File}} &
	\multicolumn{2}{l|}{Base} &
	\multicolumn{2}{l|}{Modified} &
	\multirow{2}{*}{\begin{tabular}[c]{@{}l@{}}Base/Modified \\ Time ratio\end{tabular}} &
	\multirow{2}{*}{\begin{tabular}[c]{@{}l@{}}Modified/Base \\ Cost ratio\end{tabular}} \\ \cline{2-5}
	\multicolumn{1}{|l|}{} &
	\multicolumn{1}{l|}{\begin{tabular}[c]{@{}l@{}}Time \\ elapsed\end{tabular}} &
	\multicolumn{1}{l|}{\begin{tabular}[c]{@{}l@{}}Partition\\ cost\end{tabular}} &
	\multicolumn{1}{l|}{\begin{tabular}[c]{@{}l@{}}Time\\ elapsed\end{tabular}} &
	\begin{tabular}[c]{@{}l@{}}Partition\\ cost\end{tabular} &
	&
	\\ \hline
	\multicolumn{1}{|l|}{ISPD98\_ibm01.hgr} &
	\multicolumn{1}{l|}{2311} &
	\multicolumn{1}{l|}{358} &
	\multicolumn{1}{l|}{246} &
	1598 &
	9.39 &
	4.46 \\ \hline
	\multicolumn{1}{|l|}{ISPD98\_ibm02.hgr} &
	\multicolumn{1}{l|}{2658} &
	\multicolumn{1}{l|}{520} &
	\multicolumn{1}{l|}{371} &
	512 &
	7.16 &
	0.98 \\ \hline
	\multicolumn{1}{|l|}{ISPD98\_ibm03.hgr} &
	\multicolumn{1}{l|}{2824} &
	\multicolumn{1}{l|}{3909} &
	\multicolumn{1}{l|}{429} &
	4148 &
	6.58 &
	1.06 \\ \hline
	\multicolumn{1}{|l|}{ISPD98\_ibm04.hgr} &
	\multicolumn{1}{l|}{5345} &
	\multicolumn{1}{l|}{1882} &
	\multicolumn{1}{l|}{308} &
	4096 &
	17.35 &
	2.18 \\ \hline
	\multicolumn{1}{|l|}{ISPD98\_ibm05.hgr} &
	\multicolumn{1}{l|}{8940} &
	\multicolumn{1}{l|}{3589} &
	\multicolumn{1}{l|}{709} &
	6086 &
	12.61 &
	1.70 \\ \hline
	\multicolumn{1}{|l|}{ISPD98\_ibm06.hgr} &
	\multicolumn{1}{l|}{4653} &
	\multicolumn{1}{l|}{1724} &
	\multicolumn{1}{l|}{751} &
	4799 &
	6.20 &
	2.78 \\ \hline
	\multicolumn{1}{|l|}{ISPD98\_ibm07.hgr} &
	\multicolumn{1}{l|}{7191} &
	\multicolumn{1}{l|}{3018} &
	\multicolumn{1}{l|}{1299} &
	7274 &
	5.54 &
	2.41 \\ \hline
	\multicolumn{1}{|l|}{ISPD98\_ibm08.hgr} &
	\multicolumn{1}{l|}{9761} &
	\multicolumn{1}{l|}{1382} &
	\multicolumn{1}{l|}{1360} &
	8776 &
	7.18 &
	6.35 \\ \hline
	\multicolumn{1}{|l|}{ISPD98\_ibm09.hgr} &
	\multicolumn{1}{l|}{11944} &
	\multicolumn{1}{l|}{3873} &
	\multicolumn{1}{l|}{1995} &
	9556 &
	5.99 &
	2.47 \\ \hline
	\multicolumn{1}{|l|}{ISPD98\_ibm10.hgr} &
	\multicolumn{1}{l|}{12128} &
	\multicolumn{1}{l|}{3008} &
	\multicolumn{1}{l|}{1637} &
	13001 &
	7.41 &
	4.32 \\ \hline
	\multicolumn{1}{|l|}{ISPD98\_ibm11.hgr} &
	\multicolumn{1}{l|}{10075} &
	\multicolumn{1}{l|}{8693} &
	\multicolumn{1}{l|}{2313} &
	13463 &
	4.36 &
	1.55 \\ \hline
	\multicolumn{1}{|l|}{ISPD98\_ibm12.hgr} &
	\multicolumn{1}{l|}{20845} &
	\multicolumn{1}{l|}{4622} &
	\multicolumn{1}{l|}{2959} &
	14155 &
	7.04 &
	3.06 \\ \hline
	\multicolumn{1}{|l|}{ISPD98\_ibm13.hgr} &
	\multicolumn{1}{l|}{17317} &
	\multicolumn{1}{l|}{3222} &
	\multicolumn{1}{l|}{3225} &
	16767 &
	5.37 &
	5.20 \\ \hline
	\multicolumn{1}{|l|}{ISPD98\_ibm14.hgr} &
	\multicolumn{1}{l|}{51533} &
	\multicolumn{1}{l|}{9833} &
	\multicolumn{1}{l|}{6173} &
	22643 &
	8.35 &
	2.30 \\ \hline
	\multicolumn{1}{|l|}{ISPD98\_ibm15.hgr} &
	\multicolumn{1}{l|}{41726} &
	\multicolumn{1}{l|}{7899} &
	\multicolumn{1}{l|}{6496} &
	30221 &
	6.42 &
	3.83 \\ \hline
	\multicolumn{1}{|l|}{ISPD98\_ibm16.hgr} &
	\multicolumn{1}{l|}{39586} &
	\multicolumn{1}{l|}{5539} &
	\multicolumn{1}{l|}{12863} &
	32235 &
	3.08 &
	5.82 \\ \hline
	\multicolumn{1}{|l|}{ISPD98\_ibm17.hgr} &
	\multicolumn{1}{l|}{51904} &
	\multicolumn{1}{l|}{6354} &
	\multicolumn{1}{l|}{11020} &
	40035 &
	4.71 &
	6.30 \\ \hline
	\multicolumn{1}{|l|}{ISPD98\_ibm18.hgr} &
	\multicolumn{1}{l|}{64940} &
	\multicolumn{1}{l|}{3483} &
	\multicolumn{1}{l|}{7323} &
	32570 &
	8.87 &
	9.35 \\ \hline
	\multicolumn{5}{|l|}{} &
	\textit{\textbf{7.42}} &
	\textit{\textbf{3.67}} \\ \hline
\end{tabular}
		\caption{Results of research}
		\label{tab:res}
		\twocolumn
	\end{table*}

	The reason for introducing this optimization was to check whether reduction of vertices, that are moved in FMPass(), leads to reduction of execution time. It can be seen that such an approach in most cases gives much higher convergence speed; however partition cost gets higher. It can also be seen that relative growth of partition cost w. r. t. edges number is greater than reduction of elapsed time.
	
	As a conclusion, this modification, while improving overall performance, results in a high partition cost, which may be unacceptable in some applications.
	
	\begin{thebibliography}{9}
		\bibitem{fm} C.M. Fiduccia and R.M. Mattheyses \textit{A Linear-Time Heuristic for Improving Network Partitions}, General Electric Research and Development Center.
		
	\end{thebibliography}
	
\end{document}